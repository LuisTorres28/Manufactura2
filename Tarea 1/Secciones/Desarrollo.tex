\subsection{Redes neuronales no supervisadas y redes neuronales supervisadas}
\textbf{Redes neuronales supervisadas:} Es una arquitectura de machine learning que es capaz de aprender
patrones. Los patrones que aprende dependen de la información que se le da como entrada desde una base de
datos, por lo tanto, para entrenar este tipo de redes se requiere tener un conjunto de datos ``esperados''
con los cuales la red se ajustará.\\
\textbf{Redes neuronales no supervisadas}: Son una arquitectura de machine learning capaces de aprender
patrones, pero a diferencia de las redes supervisadas, este tipo de redes no requiere de una base de datos
de entrenamiento, solamente se le indica una serie de reglas iniciales y en base a ello la red irá aprendiendo
por si sola los patrones de la iformación de entrada.
\subsection{Reinforcment learning}
El \textbf{aprendizaje reforzado} es también una técnica de machine learning que se basa en un principio
simple de recompensa. En este ambito se refiere como el ``agente'' al encargado de tomar las decisiones,
es decir, el aprendizaje lo realiza el agente basado en aprender aquellas tareas que maximicen la recompensa,
en caso de que el resultado no sea el deseado, se penaliza al agente.
\subsection{Industria 4.0}
La \textbf{industria 4.0} se considera como la cuarta revolución industrial y esta basada en la conectividad
total del proceso productivo por medio de tecnologías informáticas. Se sustena en ambientes inteligentes
interconectados, el internet de las cosas (IoT), el Big Data, control inteligente y distribuido, conectividad
en toda la cadena de suministros y proceso de producción, entre otras tecnologías de vanguardia.